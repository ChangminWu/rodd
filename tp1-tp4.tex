\documentclass[12pt,a4paper]{article} 

\usepackage{amsmath,amsfonts,amssymb,latexsym,graphicx,array}
\usepackage[french]{babel}
\usepackage[T1]{fontenc} 
\usepackage[utf8]{inputenc}
\usepackage{color}
\usepackage[a4paper, margin={3cm, 3cm}]{geometry}
\usepackage{subfigure}
\usepackage{epstopdf}
\usepackage{ragged2e}
\usepackage{tikz}
\usepackage{array}
\usepackage{multirow}

\newcommand\MyBox[2]{
  \fbox{\lower0.75cm
    \vbox to 1.7cm{\vfil
      \hbox to 1.7cm{\hfil\parbox{1.4cm}{#1\\#2}\hfil}
      \vfil}%
  }%
}
\begin{document}
%%%%%%%%%%%%%%%%%%%%%%%%%%%%%%%%%%%%%%%%%%%%%%%%%%%%%%%%%%%%%%%%%%%%%%%%%%%%%%%%%%%%%%%%%%%%%
% page titre
%%%%%%%%%%%%%%%%%%%%%%%%%%%%%%%%%%%%%%%%%%%%%%%%%%%%%%%%%%%%%%%%%%%%%%%%%%%%%%%%%%%%%%%%%%%%%
\setcounter{page}{0}


\begin{center}

% logo
\begin{minipage}[l]{.2\linewidth}
	\flushleft\includegraphics[width=4cm]{img/logo_ENSTA.png}
\end{minipage}
\begin{minipage}[r]{.35\linewidth}
	\flushright\includegraphics[width=4cm]{img/logo_CNAM.png}
\end{minipage}
\begin{minipage}[r]{.3\linewidth}
	\flushright\includegraphics[width=4cm]{img/logo_X.jpg}
\end{minipage}
\vspace{1.5cm}

% titre
\hrule \vspace{0.5cm}
\begin{LARGE}
	\textbf{RODD-TP1-TP4}\\
\end{LARGE}
\vspace{0.5cm} \hrule
\vspace{3cm}

% noms
\definecolor{carmine}{rgb}{0.59, 0.0, 0.09}
\begin{Large}\color{carmine}
	Changmin WU\\
	Ling MA\\
\end{Large}
\vspace{5cm}

% Nom écoles
\begin{large}
	\textit{Conservatoire National des Arts et Métiers\\
	--\\
	\'Ecole Polytechnique\\
	--\\
	\'Ecole Nationale Supérieure de Techniques Avancées\\}
\end{large}
\vfill

% date
\today
\end{center}

\thispagestyle{empty}
\newpage

%%%%%%%%%%%%%%%%%%%%%%%%%%%%%%%%%%%%%%%%%%%%%%%%%%%%%%%%%%%%%%%%%%%%%%%%%%%%%%%%%%%%%%%%%%%%%
%%%%%%%%%%%%%%%%%%%%%%%%%%%%%%%%%%%%%%%%%%%%%%%%%%%%%%%%%%%%%%%%%%%%%%%%%%%%%%%%%%%%%%%%%%%%%

\section{Définition du modèle}
\[
	max\qquad \sum_{m=1}^M \sum_{t=1}^T (p_t^mx_t^m + f_t^my_t^m)+ \sum_{t=1}^T h_t(s_t)
\]

\[
	s.c.\qquad -y_{ijhl}+x_{ij}+x_{hl}<=1\qquad \forall i,j,h,l \in [\![1,n]\!]
	\eqno{(1)}
\]
\[
	\qquad \qquad \qquad \qquad -y_{ijhl}<=0\qquad \forall i,j,h,l \in [\![1,n]\!]
	\eqno{(2)}
\]
\[
	\qquad -x_{ij}<=0\qquad \forall i,j\in [\![1,n]\!]
	\eqno{(3)}
\]
\[
	\qquad x_{ij}<=1\qquad \forall i,j\in [\![1,n]\!]
	\eqno{(4)}
\]
\section{Conclusion}

\centering
\begin{tikzpicture}[
                        solid/.style = {circle, draw, fill = red, minimum size = 0.3cm},
                        empty/.style = {circle, draw, fill = white, minimum size = 0.3cm}]

                        \node [empty,label=below:$x_{11}$] (A) at (1,1) {};
                        \node [empty,label=left:$x_{12}$] (B) at (1,2) {};
                        \node [empty,label=left:$x_{13}$] (C) at (1,3) {};
                        \node [empty,label=above:$x_{14}$] (D) at (1,4) {};
                        \node [empty,label=below:$x_{21}$] (E) at (2,1) {};
                        \node [empty,label=below:$x_{22}$] (F) at (2,2) {};
                        \node [empty,label=below:$x_{23}$] (G) at (2,3) {};
                         \node [empty,label=above:$x_{24}$] (H) at (2,4) {};
					 \node [empty,label=below:$x_{31}$] (I) at (3,1) {};
                        \node [empty,label=below:$x_{32}$] (J) at (3,2) {};
                        \node [empty,label=below:$x_{33}$] (K) at (3,3) {};
                        \node [empty,label=above:$x_{34}$] (L) at (3,4) {};
                        \node [empty,label=below:$x_{41}$] (M) at (4,1) {};
                        \node [empty,label=right:$x_{42}$] (N) at (4,2) {};
                        \node [empty,label=right:$x_{43}$] (O) at (4,3) {};
                         \node [empty,label=above:$x_{44}$] (P) at (4,4) {};
    \draw[line width=1pt] (A) -- (B);
    \draw[line width=1pt] (B) -- (C);
    \draw[line width=1pt] (C) -- (D);
    
    \draw[line width=1pt] (F) -- (G);
    \draw[line width=1pt] (G) -- (H);
    \draw[line width=1pt] (E) -- (F);
    
    \draw[line width=1pt] (I) -- (J);
    \draw[line width=1pt] (J) -- (K);
    \draw[line width=1pt] (K) -- (L);
    
    \draw[line width=1pt] (M) -- (N);
    \draw[line width=1pt] (N) -- (O);
    \draw[line width=1pt] (O) -- (P);
    
     \draw[line width=1pt] (A) -- (E);
    \draw[line width=1pt] (E) -- (I);
    \draw[line width=1pt] (I) -- (M);
    
     \draw[line width=1pt] (B) -- (F);
    \draw[line width=1pt] (F) -- (J);
    \draw[line width=1pt] (J) -- (N);
    
    \draw[line width=1pt] (C) -- (G);
    \draw[line width=1pt] (G) -- (K);
    \draw[line width=1pt] (K) -- (O);
    
     \draw[line width=1pt] (D) -- (H);
    \draw[line width=1pt] (H) -- (L);
    \draw[line width=1pt] (L) -- (P);
    \end{tikzpicture}

\centering    
  \noindent
\renewcommand\arraystretch{1.5}
\setlength\tabcolsep{0pt}
\begin{tabular}{c >{\bfseries}r @{\hspace{0.7em}}c @{\hspace{0.4em}}c @{\hspace{0.7em}}l}
  \multirow{10}{*}{\parbox{1.1cm}{\bfseries\raggedleft}} & 
    & \multicolumn{2}{c}{\bfseries matrice de contrainte} & \\
  && \bfseries x & \bfseries y & \bfseries \\
  &x& \MyBox{ (1)}{TU} & \MyBox{(1)}{identidé -1}&y  \\[2.4em]
  &x& \MyBox{ (2)}{0} & \MyBox{(2)}{identidé -1} &y\\
  &x& \MyBox{(3)}{identidé -1} & \MyBox{(3)}{0} &y \\[2.4em]
  &x& \MyBox{(4)}{identidé 1} & \MyBox{(4)}{0}&y\\
\end{tabular}
\newline 
\justify
Contrainte (1) pour les variables x est une grille, une grille n'est pas de circle impair, alors une grille est biparti. Par la collaire 6.1 vue en cours de PM, la matrie sommet-arête d'un graphe biparti. Alors contrainte (1) pour les variables x est bien TU. Par ailleurs, les variables y est une matrice idendité, le second membre de contrainte est un nombre entier -1. Par la propriété vue en cours de OG: Propriété 2 : si une matrice est TU, alors la juxtaposition de cette matrice avec la matrice identité est aussi une matrice TU. On a contrainte (1) est bien TU.

Contrainte (2),(3),(4) sont des matrices identités avec le seconde membre de contrainte sont un nombre entier. Ils sont TU.

Par la caractérisation de Ghouila-Houri vue en cours OG : une matrice est TU ssi tout sousensemble de colonnes peut être partitionné en deux parties telles que, sur chaque ligne, les sommes des éléments sur chacune de ces deux parties diffèrent d'au plus 1. Le matrice combinaision de contrainte (1), (2), (3), (4) peut-être partitioné en deux parties vérifié ce définition. On peut aussi calculer le det(x,y). On peut alors montrer ces contraintes sont TU.

(vi)
si on ajoute le contrainte (5) 
	$\sum -x_{ij}\leq 60,\forall i,j \in [[\![1,n]\!]$, il n'est plus TU car il y a au moin 60 de -1 dans chaque row. Si on l'ajoute ce matrice dans la matrice obtenu dans (iii), il n'est plus TU.

\end{document}