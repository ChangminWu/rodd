\documentclass[a4paper]{article}

\usepackage[english]{babel}
\usepackage[utf8]{inputenc}
\usepackage{amsmath}
\usepackage{graphicx}
\usepackage[colorinlistoftodos]{todonotes}
\usepackage{amsmath}  
\newtheorem{theorem}{Theorem}  
\newtheorem{definition}{Definition}  
\newtheorem{lemma}{Lemma}  
\newtheorem{proposition}{Proposition}
\newtheorem{proof}{Proof}[section] 
\usepackage{indentfirst}
\usepackage{bbm}
\usepackage{multirow}
\advance\day by -2

\title{Problème de planification culturale durable \footnote{Ce rapport est destiné au TP7 de cours Recherche Opérationnelle et Développement Durable du programme MPRO encadré par Professeur Agnès Plateau sur le sujet de la planification culturale.}}

\author{Ling \textsc{Ma} and Changmin \textsc{Wu}}
\date{\today}

\begin{document}
\maketitle
\tableofcontents

\section{Quelques notations}
\begin{itemize}
    \item $\mathcal{T}$: $\{1, \ldots, T\}$ le horizon de planification.
    \item $\mathcal{P}$: $\{1, \ldots, P\}$ ensemble des parcelles utilisables.
    \item $s(t)$: $s(t)=1$ temps impair; $s(t)=2$ temps pair.
    \item $C(s(t))$: ensemble des cultures cultivables au temps $t$.
    \item $D_{j,t}$: demande en tonnes de culture $j$ au temps $t$.
    \item $(l,a,j)$: état d'une parcelle où
    \begin{itemize}
        \item $l$: le nombre de temps consécutifs de jachère.
        \item $a$: le nombre de temps consécutifs de culture (si $j \neq 0$).
        \item $j$: $C \cup \{0\}$ où $C$ l'ensemble des cultures cultivables et $0$ la jachère.
    \end{itemize}
    \item $(l',a',i) -> (l,a,j)$: une transition d'une parcelle d'état $(l',a',i)$ à $(l,a,j)$ qui est associé avec un rendement $\text{REND}(l,a,i,j)$.
\end{itemize}

\section{Une représentation du graphe}
Figure \ref{fig:diam} présente un tel graphe dont chaque sommet désigne un état possible de la forme $(l,a,j)$. Un chemin qui commence par $(2,0,0)$ (et de longueur $5$) est donc une rotation de $T=5$ sur la parcelle $p$, par exemple, le chemin $(2,0,0)->(2,1,R)->(2,2,H)->(1,0,0)->(1,1,H)->(1,0,0)$. 
\begin{figure}[h!t]
    \centering
    \includegraphics[width=\textwidth]{diagram.pdf}
    \caption{Graphe $G$: les arcs rouges correspondent à une rotation possible; $R$ est \textit{riz} et $H$ est \textit{haricot}}
    \label{fig:diam}
\end{figure}
Notons que chaque arc désigne une transition $(l',a',i) -> (l,a,j)$, on peut donc associer chaque arc un rendement $\text{REND}_{arc} = \text{REND}(l,arc,i,j)$ où $arc = (l',a',i) -> (l,a,j)$

\section{Modélisation PLNE}
Étant donnée la représentation du graphe, on peut modéliser ce problème de planification comme $P$ problème de flot sur $P$ graphe $G$ qui satisfit une contrainte globale. 
\begin{equation*}
    (P1) \left\{ 
    \begin{aligned}
    \min\quad        & \sum_{p\in P}\sum_{t=1}\sum_{arc \in Arcs^{-}_{(2,0,0)}} x_{p,t,arc}   \\
    \text{s.t.\quad} & \sum_{p\in P}\sum_{arc \in Arcs^{-}_{j}} \text{REND}_{arc} x_{p,t,arc} \geq D_{j,t} & & \forall t\in \mathcal{T}, j \in C(s(t)) \\
                     & \sum_{arc \in Arcs^{-}_{v}} x_{p,t,arc} = \sum_{b \in Arcs^{+}_{v}} x_{p,t+1,b} & & \forall t \in \mathcal{T}\backslash\{1,T\}, p \in \mathcal{P}, v\in V(G)\\
                     & \sum_{arc \in Arcs^{-}_{(2,0,0)}} x_{p,1,arc} \leq 1 & &  \forall p \in \mathcal{P} \\
                     & \sum_{arc \not\in Arcs^{-}_{(2,0,0)}} x_{p,1,arc} = 0 & &  \forall p \in \mathcal{P} \\
                     & x_{p,t,arc} \in \{0,1\} & & \forall t \in \mathcal{T}, p \in \mathcal{P}, arc \in Arcs
  \end{aligned}
\right.
\end{equation*}
où $x_{p,t,a}$ désigne si la transition représentée par $a$ s'effectue au temps $t$ sur la parcelle $p$. $Arc^{-}_{j}$ désigne tous arcs de $G$ incident à $j$ et $Arc^{+}_{j}$ les arcs émergent de $j$. La première contrainte est la contrainte de la demande. La deuxième contrainte est celui de la conservation du flot et les deux prochaines sont les contraintes d'état initial (tout chemin commence de $(2,0,0)$).

\section{Résultat}
Voir tableau \ref{tab:res}.
\begin{table}
    \centering
    \resizebox{1 \textwidth}{!}{
    \begin{tabular}{c|c|c}
    \hline
    Temps de calcul & Nbr Noeuds développés & Nbr Parcelles Cultivées \\
    \hline
    $2.47$s & $1344$ & $19$ \\
    \hline
    \end{tabular}}
    \caption{Information de la solution trouvée}
    \label{tab:res}
\end{table}

\section{Une ré-formulation}
Supposons que les rotations $r$ sont énumérable, et elle est représentée par une suite de transitions (arcs). on a
\begin{equation*}
    (P2) \left\{ 
    \begin{aligned}
    \min\quad        & \sum_{r \in R} x_{r}   \\
    \text{s.t.\quad} & \sum_{r \in R} A_{j,t,r} x_{r} \geq D_{j,t} & & \forall t\in \mathcal{T}, j \in C(s(t))\\
                     & x_r \in \mathbbm{Z}_+            & & \forall r
  \end{aligned}
\right.
\end{equation*}
où $r_t$ est la transition de la rotation $r$ au temps $t$, $A_{j,t,r} = \text{REND}_{r_t}$ si $r_t \in Arcs^{-}_{j}$ et $0$ autrement. 

\section{Génération de colonnes}

We define a colonne for the path satifait all the demand. For example in Quesiton 1,a solution chemin P: $(2,0,0)->(2,1,R)->(2,2,H)->(1,0,0)->(1,1,H)->(1,0,0)$. \\
Colonne R correspondant à A is [(2,0,0),(2,1,R),(2,2,H),(1,0,0),(1,1,H),(1,0,0)]. Each colonne has T+1 variable. For example, this colonne R has 6 variable.
\\

		Algorithem of generation colonne:\\

		Create initial columns;
		
			Repeat:
			
					 Solve master problem find $dual[i]$;
					 
					 Solve subproblem;
					 
					 Add new column to master problem;
					 
	   		 Until $z_{sub} \geq 0$

We start from a init solution which can satfifait all the demand of different of time periode. We define a variable $choose_r$, if $choose_r$=1, that means we choose the colonne, if $choose_r$=0;
that means we won't choose the chemin.With the contraintes,demande of every time t is satisfied. Here we doesn't show the model of the contraintes in details.
\begin{equation*}
    (master) \left\{ 
    \begin{aligned}
    \min\quad   choose_{r}*x_{r} , \qquad r \in Chemins \\
  \end{aligned}
\right.
\end{equation*}

We dualize $newcollne_{j,t}$ contrainte of the master problem, and put them into dual variable.  Every time when we solve the slave problem, the solution z that we get is a new solution, or we say it, a new chemin réalisable for the master problem.

\begin{equation*}
    (slave) \left\{ 
    \begin{aligned}
    \min\quad  & 1-\sum_{j,t} newcolonne_{j,t}\cdot dual_{j,t}\\
  \end{aligned}
\right.
\end{equation*}

As the process of algo, we will have more and more colonnes to make a choice in them. Everytime we solve one time slave problem, and them we solve master problem again. If the coût réduit is négatif, we can keep on generate new colonne, until the coût réduit $\geq$0, we stop generate new colonne, and the current solution of master problem is optimal.

\end{document}
